Welcome! If you are reading this document, you have probably been appointed as the next TA to run ECE210 Honors. As the name suggests, this is a transition document written for you to provide useful information and to help you get up to speed.

This document was initiated by Kaidong Peng, a Spring 2018 graduate who was in charge of ECE210 Honors from Spring 2016 to Fall 2017. When Kaidong took over this job from his predecessor, he had encountered many difficulties familarizing himself with various administrative tasks. As a result, he spent a lot of time and efforts figuring out these tasks, which in his opinion could totally be avoided if there is a properly written guideline or transition document for him to refer to. Fortunately, his predecessor stayed at University of Illinois for graduate study and thus he was able to get a lot of help. Nevertheless, he felt the need of a properly written transition document and was therefore motivated to creat this transition document to help his successors.

As you probably know, ECE210 Honors is achieved by the collective efforts of previous ECE students: the idea of this class originated from a student and the class is entirely run by students as well. According to Professor Erhan Kudeki, who is the course director of ECE210 and the deparment associate head for undergraduate affairs by the time Kaidong created this document, some students came to ask Professor Kudeki if there is a clas they could learn useful scientific computation tools from. Remember this happened in the good old days when ECE210 Honors does not exist. I don't know what was exactly Professor Kudeki\rq s answer to them, but later another student came to him and volunteered to run a session to teach students MATLAB. Professor Kudeki supported him and this is how ECE210 Honors started.

Initially ECE210 Honors only taught students MATLAB, but later with the increasing popularity of Python and possibly with Professor Kudeki\rq s enthusiasm to Python as well, Jupyter (previously known as iPython) was added into to the syllabus. Keep in mind that the goal of this class is to teach students useful scientific computation tools and basic digital signal processing, so feel free to add new things you consider helpful and remove things you consider outdated. That being said, if one day MATLAB becomes completely unpopular or extremely expensive that even EWS machines no longer offer them, you should consider removing those MATLAB content and add other useful things into this class. For instance, if there is a new emerging language or tool, say MightyKitten (I made it up), you should seriously consider adding it into the syllabus if it is highly useful and beneficial to students.

Okay, so what do you need to know in order to become a successful TA for ECE210 Honors? First thing first: you have to have a great understanding in the tools you are about to teach. This is obvious: if you are struggling with MATLAB and Python yourself, and you are trying to teach others on how to use them and use them well, students will have no confidence in you and you probably won\rq t be a very good TA. An excellent understanding in signal processing is also very critical: you want to teach students not only the correct way to do things but the correct way to think as well.

But these are not the end of story. Besides these technical knowledge and understanding, you need to know various administrative procedures, such as room reservation, software requests, registration logistics, grades reporting, and much more to run this class by yourself. And yes, you need to come up with a good course flow to optmize students\rq \
 learning experience. This sounds like a very daunting task, isn\rq t it? No worries, I\rq ve got you covered. Again, this is exactly why this document is here for you to read before you set out to teach ECE210 Honors in the upcoming semester. After reading through this document, you should have enough knowledge necessary to organize and run this class smoothly. In turn, I ask you a favor to make improvements to this document in future when you have learned something new: anything you add will help your successors significantly, and I am pretty sure they will greatly appreciate your contributions.

Now fasten your seat belt. Let\rq s get to know the different aspects of this job one by one in great details.